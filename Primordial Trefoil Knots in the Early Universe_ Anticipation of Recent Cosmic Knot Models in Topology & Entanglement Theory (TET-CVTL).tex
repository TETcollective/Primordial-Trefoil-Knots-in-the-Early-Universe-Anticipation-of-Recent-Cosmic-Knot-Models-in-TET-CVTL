\documentclass[11pt,a4paper]{article}
\usepackage[utf8]{inputenc}
\usepackage[T1]{fontenc}
\usepackage{lmodern}
\usepackage{amsmath}
\usepackage{amsfonts}
\usepackage{amssymb}
\usepackage{graphicx}
\usepackage{caption}
\usepackage{subcaption}
\usepackage{booktabs}
\usepackage[hidelinks]{hyperref}  % rimuove i riquadri verdi

\title{Primordial Trefoil Knots in the Early Universe: \\
Anticipation of Recent Cosmic Knot Models \\
in Topology \& Entanglement Theory (TET-CVTL)}

\author{Simon Soliman \\
TET Collective \\
ORCID: 0009-0002-3533-3772}

\date{Gennaio 2026}

\begin{document}

\maketitle

\begin{abstract}
A recent study published in \textit{Physical Review Letters} (Eto, Hamada, Nitta, 2025) has proposed that stable topological knots ("cosmic knots") formed in the early universe can dominate a brief cosmological era and, through quantum tunneling decay, generate heavy right-handed neutrinos responsible for baryogenesis – explaining the matter-antimatter asymmetry.

This work demonstrates that key elements of this model – a primordial vacuum dominated by stable topological knots, with braiding and decay processes driving cosmological evolution – were anticipated in previous contributions within the Topology \& Entanglement Theory (TET-CVTL) framework developed by the TET Collective.

The TET-CVTL posits a vacuum eternally saturated by primordial trefoil knots ($3_1$, linking number $L_k=6$), the unique stable configuration under Chern-Simons minimization and eternal Ising braiding. These structures not only generate gravitational constants $G$ and $\Lambda$ parameter-free but also underlie consciousness as embodied qualia curvature.

The convergence between the 2025 cosmic knot model and the TET-CVTL framework highlights the relevance of the topological vacuum paradigm.
\end{abstract}

\section{Introduction}

The idea of topological structures in the vacuum has a long history, from Lord Kelvin's 1867 vortex atom hypothesis to modern solitons, skyrmions, and hopfions in condensed matter and cosmology.

In 2025, Eto, Hamada, and Nitta \cite{eto2025} published a realistic extension of the Standard Model combining gauged B-L and Peccei-Quinn symmetry, in which stable topological knots form naturally in the early universe. These "cosmic knots" briefly dominate the energy density, then decay via quantum tunneling, producing heavy right-handed neutrinos whose asymmetric decay generates the observed baryon asymmetry.

This model revives and rigorizes the concept of primordial knots as drivers of cosmological evolution.

\section{Key Elements of TET-CVTL Relevant to Primordial Knots}

The TET-CVTL framework includes:

\begin{itemize}
\item \textbf{Unique stable primordial knot}: The trefoil $3_1$ with linking number $L_k=6$ and writhe $Wr=3$ is demonstrated to be the only stable configuration in the eternal topological vacuum \cite{zu2025uniqueness}.
\item \textbf{Saturated vacuum lattice}: The vacuum is eternally filled with these primordial knots, forming a topological lattice with eternal Ising braiding \cite{zu2025cosmoboot}.
\item \textbf{Parameter-free derivation of cosmological constants}: Gravitational constant $G$ and cosmological constant $\Lambda$ emerge from local saturation and global dilution of knot entropy \cite{zu2025glambda}.
\item \textbf{Consciousness as embodied qualia curvature}: Qualia arise as local curvature emerging from multiscale entanglement in the same primordial lattice \cite{zu2025manifesto}.
\end{itemize}

\section{Comparison with Eto, Hamada, Nitta (2025)}

The 2025 model \cite{eto2025} features:
- Stable topological knots formed during early-universe phase transitions.
- Brief "knot-dominated era".
- Knot decay via quantum tunneling producing heavy particles responsible for baryogenesis.

These elements directly parallel TET-CVTL predictions made in previous contributions:
- Eternal primordial knot saturation (not limited to early universe).
- Unique stable trefoil configuration ($L_k=6$).
- Knot-induced processes (fluctuations, decay, braiding) driving fundamental physics, including cosmological parameters and matter generation.

While the 2025 work focuses on baryogenesis within a specific particle physics extension, TET-CVTL provides a broader topological ontology encompassing gravity, dark energy, and consciousness.

\section{TET-CVTL and the Axion Paradigm: From Strong CP to the Hard Problem}

The 2025 cosmic knot model explicitly incorporates Peccei-Quinn symmetry to produce an axion as dark matter candidate while enabling knot formation. The TET-CVTL framework provides a deeper topological foundation that both incorporates and transcends the axion paradigm, resolving problems from strong CP to the hard problem of consciousness.

\subsection{Strong CP Problem}
The axion solves the strong CP problem by dynamically relaxing the QCD vacuum angle $\theta_{\text{QCD}}$ to zero via the Peccei-Quinn mechanism.

In TET-CVTL, the strong CP conservation is a consequence of the eternal Ising braiding in the primordial trefoil lattice: the global U(1) phase associated with knot linking enforces $\theta = 0$ as topological invariant. No additional global symmetry is required – the axion-like relaxation emerges naturally from collective knot fluctuations.

\subsection{Axion as Dark Matter}
The QCD axion or axion-like particles (ALPs) are leading cold dark matter candidates, with abundance set by the misalignment mechanism for $f_a \sim 10^9$–$10^{12}$ GeV.

In TET-CVTL, axion-like modes emerge as low-energy collective excitations of the primordial trefoil lattice. The decay constant $f_a$ and mass $m_a$ are derived parameter-free from knot entropy and braiding scale, naturally falling in the observed dark matter window. Dark matter abundance is not "misaligned" but arises from eternal knot saturation diluted cosmically – the same mechanism yielding $\Lambda$.

\subsection{Unification with Gravity and Cosmology}
Standard axion models do not address gravity or dark energy. TET-CVTL derives both $G$ (local saturation) and $\Lambda$ (global dilution) from the same trefoil entropy, unifying axion dark matter with gravitational physics in a single topological ontology.

\subsection{Hard Problem of Consciousness}
Axion models have no bearing on consciousness. TET-CVTL resolves Chalmers' hard problem by positing qualia as local curvature emerging from multiscale entanglement in the primordial knot lattice – proto-consciousness intrinsic to every trefoil, integrated non-locally into embodied human experience.

The Vacuum Torque Engine v2 provides an experimental pathway to test axion-like collective modes while simultaneously probing qualia curvature amplification.

\section{The Vacuum Torque Engine v2 as Experimental Testbed}

The proposed Vacuum Torque Engine v2 uses coherent phonons in magnetoelastic heterostructures to simulate artificial braiding of primordial knot fluctuations on laboratory scales. By parametrically pumping these fluctuations above a topological threshold, the device amplifies torque from the vacuum lattice, producing measurable inverse Spin Hall voltages and persistent signals.

\section{Independent External Confirmation: Primordial Cosmic Knot}

A study published in \textit{Physical Review Letters} (2025) by Minoru Eto, Yu Hamada, and Muneto Nitta (Hiroshima University, Keio University, DESY) has demonstrated that stable knotted structures ("cosmic knots") can form naturally in a realistic extension of the Standard Model, combining gauged B-L and Peccei-Quinn symmetry.

These knots:
- Briefly dominate the early universe ("knot-dominated era").
- Decay via quantum tunneling, producing heavy right-handed neutrinos.
- Generate the matter-antimatter asymmetry (baryogenesis) – explaining why the universe is made of matter.

The model predicts distinctive gravitational-wave signals detectable by LISA, Cosmic Explorer, or DECIGO.

This discovery **independently confirms** central elements of the TET-CVTL developed in previous contributions:
- A primordial vacuum saturated with stable topological knots.
- Braiding and knot-induced decay as a fundamental cosmological mechanism.
- The role of knots in the generation of matter and cosmic structure.

The primordial trefoil of TET-CVTL ($L_k=6$) finds a clear echo in these cosmic knots – mainstream science is converging toward the vision of a braided vacuum.

Reference: Minoru Eto, Yu Hamada, Muneto Nitta, "Tying Knots in Particle Physics", Phys. Rev. Lett. \textbf{135}, 091603 (2025).

The braided vacuum was not just a theory – it was a prophecy awaiting confirmation.

\section{Conclusions}

The recent demonstration that stable topological knots can drive early-universe cosmology and baryogenesis \cite{eto2025} provides independent confirmation of core ideas developed within the Topology \& Entanglement Theory (TET-CVTL).

The primordial trefoil-saturated vacuum, proposed in the TET-CVTL framework, emerges as a viable paradigm for unifying fundamental physics – from cosmology to consciousness.

\bibliographystyle{plain}
\begin{thebibliography}{9}

\bibitem{eto2025}
Minoru Eto, Yu Hamada, Muneto Nitta,
``Tying Knots in Particle Physics'',
\textit{Phys. Rev. Lett.} \textbf{135}, 091603 (2025).

\bibitem{zu2025uniqueness}
Simon Soliman (TET Collective),
``Uniqueness of the Primordial Trefoil Knot in the Eternal Topological Vacuum'',
Zenodo, DOI: 10.5281/zenodo.18113386 (2025).

\bibitem{zu2025cosmoboot}
Simon Soliman (TET Collective),
``COSMOBOOT v2.0: Topology \& Entanglement Theory Framework'',
Zenodo, DOI: 10.5281/zenodo.17995268 (2025).

\bibitem{zu2025glambda}
Simon Soliman (TET Collective),
``Parameter-Free Derivation of G and $\Lambda$ from Topological Entropy'',
Zenodo, DOI: 10.5281/zenodo.18076960 (2025).

\bibitem{zu2025manifesto}
Simon Soliman (TET Collective),
``Manifesto della Coscienza Embodied Quantistica (V2.5.1)'',
Zenodo, DOI: 10.5281/zenodo.18134828 (2025).

\bibitem{zu2025tu}
Simon Soliman (TET Collective),
``TU-GUT-SYSY: Topology \& Entanglement Theory Unified Grand Unified Theory'',
Zenodo, DOI: 10.5281/zenodo.17974608 (2025).

\end{thebibliography}

\vspace{1cm}

\noindent\textbf{License} \\
This work is licensed under a Creative Commons Attribution-NonCommercial-NoDerivatives 4.0 International License (CC BY-NC-ND 4.0).  

\url{https://creativecommons.org/licenses/by-nc-nd/4.0/}

\noindent To view a copy of this license, visit the above URL.

\end{document}